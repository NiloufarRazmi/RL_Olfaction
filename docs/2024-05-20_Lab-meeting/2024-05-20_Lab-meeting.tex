% Created 2024-05-20 Mon 08:18
% Intended LaTeX compiler: lualatex
\documentclass[bigger]{beamer}
\usepackage{amsmath}
\usepackage{fontspec}
\usepackage{graphicx}
\usepackage{longtable}
\usepackage{wrapfig}
\usepackage{rotating}
\usepackage[normalem]{ulem}
\usepackage{capt-of}
\usepackage{hyperref}
\usetheme[progressbar=foot, sectionpage=none, numbering=fraction]{metropolis}
\usepackage{tikz}
\usepackage{booktabs}
\usepackage{adjustbox}
\usepackage{diagbox}
\usepackage{latexcolors}
\usetikzlibrary{automata, positioning, arrows, arrows.meta}
\usepackage{diagbox}
\usepackage{dsfont}
\usepackage{amsmath}
\usepackage{fontawesome}
\usepackage{pgfgantt}
\usepackage[ruled]{algorithm2e}
\usepackage[absolute, overlay]{textpos}
\definecolor{RedBrown}{RGB}{192, 4, 4} \setbeamercolor{progress bar}{fg=RedBrown} \setbeamercolor{title separator}{fg=RedBrown}
\setbeamercolor{progress bar in head/foot}{fg=RedBrown} \setbeamercolor{progress bar in section page}{fg=RedBrown} \setbeamercolor{alerted text}{fg=RedBrown}
\pretocmd{\tableofcontents}{\thispagestyle{empty}}{}{}
\addtocounter{framenumber}{-1}
\usepackage{listings}
\usepackage{xcolor}
\definecolor{codegreen}{rgb}{0,0.6,0}
\definecolor{codegray}{rgb}{0.5,0.5,0.5}
\definecolor{codepurple}{rgb}{0.58,0,0.82}
\definecolor{backcolour}{HTML}{f0f0f0}
\lstdefinestyle{mystyle}{
backgroundcolor=\color{backcolour},
commentstyle=\color{codegreen},
keywordstyle=\color{magenta},
numberstyle=\tiny\color{codegray},
stringstyle=\color{codepurple},
basicstyle=\ttfamily,
breakatwhitespace=false,
breaklines=true,
captionpos=b,
keepspaces=true,
numbers=none,
numbersep=5pt,
showspaces=false,
showstringspaces=false,
showtabs=false,
tabsize=2
}
\lstset{style=mystyle}
\usetheme{default}
\author{Andrea Pierré}
\date{May 20, 2024}
\title{Research plan}
\subtitle{Lab meeting}
\institute{Brown University}
\titlegraphic{\hfill\includegraphics[height=1.5cm]{img/Brown Logo_2016_2 Color Process ST_1300.png}}
\setbeamercovered{transparent=10}
\setbeamertemplate{section in toc}[sections numbered]
\AtBeginSection[]{\begin{frame}[plain, noframenumbering]{Outline}    \setbeamertemplate{section in toc}[sections numbered]\setbeamertemplate{subsection in toc}[subsections numbered]\tableofcontents[currentsection, currentsubsection]\end{frame}}
\AtBeginSubsection[]{\begin{frame}[plain, noframenumbering]{Outline}\setbeamertemplate{section in toc}[sections numbered]\setbeamertemplate{subsection in toc}[subsections numbered]\tableofcontents[currentsection,currentsubsection]\end{frame}}
\hypersetup{
 pdfauthor={Andrea Pierré},
 pdftitle={Research plan},
 pdfkeywords={},
 pdfsubject={},
 pdfcreator={Emacs 29.3 (Org mode 9.7)}, 
 pdflang={English}}
\begin{document}

\maketitle
\begin{frame}[plain]{Outline}
\tableofcontents
\end{frame}

\section{Context \faicon{question-circle}}
\label{sec:orga4882f3}

\begin{frame}[<+->][label={sec:orgacd0703}]{Why modeling?}
\begin{itemize}
\item Posit: You understand a system if you can simulate it
\end{itemize}
\begin{quote}
What I cannot create, I do not understand. --Richard Feynman
\end{quote}
\begin{itemize}
\item If you have a good enough model you may uncover mechanisms that explain a phenomena
\begin{itemize}
\item Without a model \(\to\) you're limited to describe the \alert{how}
\item With a model \(\to\) you may be able to explain the \alert{why}
\end{itemize}
\item Test hypothesis
\item Abstraction of the system: makes you think of the parameters/inputs/outputs
\item Find out what is needed to reproduce experimental results, what explains those results
\end{itemize}
\end{frame}
\begin{frame}[label={sec:org13f81d1}]{Recap of previous episodes: it's converging! \faicon{smile-o}}
\begin{center}
\includegraphics[width=\linewidth]{img/steps-and-rewards.png}
\end{center}
\end{frame}
\begin{frame}[<+->][label={sec:org0595c4b}]{Why is it converging now?}
\begin{itemize}
\item Lights cues in the state?
\item Start training once replay buffer is full (5000 transitions) instead of when there are enough transitions for a batch (32 transitions)
\item Soft update of the networks weights (instead of sharp transition)
\item Huber loss instead of mean squared error \(\to\) should be less sensible to outliers
\item \alert{Remove ReLU on output layer!}
\end{itemize}
\end{frame}
\begin{frame}[label={sec:org91b2a66}]{What did it learn?}
\begin{center}
\includegraphics[width=\linewidth]{img/policy.png}
\end{center}
\end{frame}
\begin{frame}[<+->][label={sec:org469cde1}]{What do we want to know?}
\begin{itemize}
\item Understand what the network learns
\(\to\) What \alert{function} does it learns?
\item How the constrains of the task affect learning \& the representations learned?
\item Does the network learn something related to the real neurons? (million \faicon{dollar}\faicon{dollar}\faicon{dollar} question)
\end{itemize}
\end{frame}
\begin{frame}[<+->][label={sec:orge437b65}]{Compositional hypothesis}
\begin{itemize}
\item Does the network learn a generalizable policy? \(\to\)~How to test it?
\begin{itemize}
\item From indexed locations (current) to coordinate system
\item Merged actions space
\end{itemize}
\end{itemize}
\end{frame}
\begin{frame}[label={sec:org9e330bc}]{Compositional hypothesis}
\begin{center}
\includegraphics[width=.9\linewidth]{img/env_new-triangle-task.drawio.png}
\end{center}
\end{frame}
\begin{frame}[label={sec:org40b847a}]{Implementation}
\begin{center}
\includegraphics[width=.9\linewidth]{img/nn.drawio.png}
\end{center}
\end{frame}
\begin{frame}[label={sec:orge31336d}]{Example episode}
\begin{center}
\includegraphics[width=.9\linewidth]{img/env_new-steps.drawio.png}
\end{center}
\end{frame}
\section{Experiments \& expected results \faicon{flask} \faicon{area-chart}}
\label{sec:org30d62a7}
\begin{frame}[<+->][label={sec:orgc382da5}]{1) How training impacts the representations learned?}
\begin{itemize}
\item Feed both coordinates information (Cartesian \& polar) to the input layer (+ merge actions spaces in a common one)
\item Train on left/right task \(\to\) we expect the weights are close to zero on Cartesian representation?
\item Train on east/west task \(\to\) we expect the weights are close to zero on polar representation?
\end{itemize}
\end{frame}
\begin{frame}[label={sec:org1a34467}]{1) How training impacts the representations learned?}
\begin{center}
\includegraphics[width=.9\linewidth]{img/exp1-weights-heatmap.png}
\end{center}
\end{frame}
\begin{frame}[<+->][label={sec:orgfd15c54}]{2) Does the network learn a coordinate system?}
\begin{enumerate}
\item After training, move the population of agents in a translated coordinate system
\(\to\) we expect the population of agents to be able to solve the task with zero shot learning
\item Train with both coordinates information (Cartesian \& polar), after training feed incorrect polar angles
\begin{itemize}
\item On the left/right task \(\to\) we expect the population of agents still solves the task consistently
\item On the east/west task \(\to\) we expect the network won't converge to a stable policy (i.e all the agents don't solve the task consistently)
\end{itemize}
\end{enumerate}
\end{frame}
\begin{frame}[label={sec:org154455c}]{2) Does the network learn a coordinate system?}
\begin{center}
\includegraphics[width=.9\linewidth]{img/exp2-boxplot.png}
\end{center}
\end{frame}
\begin{frame}[label={sec:orge0d12be}]{2) Does the network learn a coordinate system?}
\begin{center}
\includegraphics[width=.9\linewidth]{img/angles-to-activations.png}
\end{center}
\end{frame}
\section{Roadmap \faicon{road}}
\label{sec:orge6da991}
\begin{frame}[label={sec:orga6192dd}]{Experiments table}
\scriptsize
\begin{center}
\begin{tabular}{lrr}
Experiment & Agents & Training estimation [hours]\\
\hline
left/right Cartesian coordinates from center arena & 20 & 6\\
left/right Cartesian coordinates from 3 ports & 20 & 6\\
east/west polar coordinates from center arena & 20 & 6\\
east/west polar coordinates from 3 ports & 20 & 6\\
\hline
No translation & 20 & 6\\
Cartesian translated & 20 & 6\\
Polar translated & 20 & 6\\
left/right correct angle & 20 & 6\\
left/right incorrect angle & 20 & 6\\
east/west correct angle & 20 & 6\\
east/west incorrect angle & 20 & 6\\
\hline
Total & 220 & 66\\
\end{tabular}
\end{center}
\end{frame}
\begin{frame}[<+->][label={sec:orgd909bb6}]{Milestones/how to get there}
\begin{enumerate}
\item Rewrite the environment(s) \faicon{star}\faicon{star}\faicon{star-o}
\begin{enumerate}
\item Code logic for new environment \colorbox{black!15!white}{[$\textasciitilde$1 week]}
\item Check everything works as expected (unit testing) \colorbox{black!15!white}{[$\textasciitilde$1 week]}
\item Bugs? \colorbox{black!15!white}{[$\textasciitilde$1 week]}
\item Baseline training on new environment (convergence, hyperparameter tweaking, etc.) \faicon{star}\faicon{star}\faicon{star}\faicon{warning} \colorbox{black!15!white}{[1 week -- 1 month]}
\end{enumerate}
\item Experiments
\begin{enumerate}
\item Task code \faicon{star}\faicon{star-o}\faicon{star-o} \colorbox{black!15!white}{[$\textasciitilde$1 week]}
\item Training \faicon{star}\faicon{star}\faicon{star-o}\faicon{warning} \colorbox{black!15!white}{[$\textasciitilde$2 week]}
\item Analysis code \faicon{star}\faicon{star}\faicon{star-o} \colorbox{black!15!white}{[$\textasciitilde$2 week]}
\end{enumerate}
\end{enumerate}
\end{frame}
\begin{frame}[label={sec:orgb744090}]{Planning}
% \begin{figure}%[htb]
    % \centering
    \begin{adjustbox}{max width=\columnwidth, keepaspectratio}
    \begin{ganttchart}[
        y unit title=1.5em,
        y unit chart=1.3em,
        x unit=0.4em,
        vgrid,
        time slot format=isodate,
        time slot unit=day,
        title/.append style={draw=none, fill=white!20!black},
        title label font=\sffamily\bfseries\color{white},
        title left shift=.05,
        title right shift=-.05,
        title height=1,
        bar/.append style={draw=none, fill=white!50!black},
        bar height=.6,
        bar label font=\normalsize\color{black!50},
        group right shift=0,
        group top shift=.6,
        group height=.2,
        group peaks height=.3,
        bar incomplete/.append style={fill=Maroon},
        link bulge=2,
        calendar week text=W~\currentweek,
        % link tolerance=.1,
        % link/.style={-},
        ]{2024-05-13}{2024-09-15}
        \gantttitlecalendar{year, month=shortname, week=20} \\

        \ganttgroup{New baseline}{2024-05-27}{2024-07-07} \\
        \ganttbar{New environment(s) logic}{2024-05-27}{2024-06-02} \\
        \ganttlinkedbar{Unit testing}{2024-06-03}{2024-06-09} \\
        \ganttlinkedbar{Debugging?}{2024-06-10}{2024-06-16} \\
        \ganttlinkedbar{New baseline training \faicon{warning}}{2024-06-17}{2024-07-07} \\

        \ganttgroup{Experiments}{2024-07-22}{2024-09-08} \\
        \ganttbar{Tasks code}{2024-07-22}{2024-07-28} \\
        \ganttlinkedbar{Training \faicon{warning}}{2024-08-05}{2024-08-25} \\
        \ganttlinkedbar{Analysis}{2024-08-26}{2024-09-08} \\

        \ganttgroup{Misc.}{2024-05-20}{2024-08-04} \\
        \ganttbar{NWB paper}{2024-05-20}{2024-05-26} \\
        \ganttbar{DL+RL summer school}{2024-07-08}{2024-07-17} \\
        \ganttbar{Holidays}{2024-07-29}{2024-08-04} \\
    \end{ganttchart}
    \end{adjustbox}
% \end{figure}
\end{frame}
\begin{frame}[label={sec:orgdaa62a7},standout]{~}
Thanks!
\end{frame}
\end{document}
